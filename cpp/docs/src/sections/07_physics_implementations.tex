% ============================================================
% Section 07 - Physics Implementations
% ============================================================

\section{Physics Implementations}
\label{sec:physics_implementations}


% Data Structure Descriptions
\subsection{Data Structures}
\begin{description}
  \item[\texttt{\textbf{Bond}}]
    Stores information about a single bond including the indices of its lattice
    sites and its size. \\
    \textbf{Data Members}
    \begin{itemize}
      \item \texttt{int numSites}
        The number of sites in the bond.
      \item \texttt{std::set<int> indices}
        The indices of lattice sites in the bond.
        This structure works for one dimension and will need to be updated for
        three dimensions.
        For complicated geometries, the approach to storing the lattice sites 
        for each bond will need to be revised.
    \end{itemize}
    \textbf{Methods}
    \begin{itemize}
      \item \texttt{Bond(const std::set<int>\& indices)}
        This is the constructor for the Bond class.
        The indices input is a set of integers that represent the indices of the
        lattice that are participants in the bond. 
        These lattice neighbors must be nearest neighbors. 
        Additional considerations must be made once three dimensions and more 
        complicated geometries are introduced.
      \item \texttt{int getNumSites() const}
        Returns the number of sites in the bond. 
      \item \texttt{bool operator==(const Bond\& other) const}
        This overloaded equality operator tests whether two Bond objects are  
        equivalent by determining whether their indices arrays are equal.
      \item \texttt{bool operator!=(const Bond\& other) const}
        This overloaded operator uses the overloaded equaulity operator to 
        determine inequality.
      \item \texttt{friend std::ostream\& operator<<(std::ostream\& os, const Bond\& bond)}
        This overloaded operator prints bond data out in a neat, easy to read 
        format.
    \end{itemize}
  \item[\texttt{\textbf{Lattice}}]
    Stores information about the lattice including the spatial positions of 
    lattice sites and the number of sites.
    Once higher dimensions are supported, numSites will need to be a vector of
    integers, and sites will need to be a three dimensional vector of floats. \\
    \textbf{Data Members}
    \begin{itemize}
      \item \texttt{int numSites}
        The number of sites in the lattice.
      \item \texttt{std::vector<float> sites}
        The spatial positions of each site in one dimension. 
        Only supports a single dimension at the moment.
    \end{itemize}
    \textbf{Methods}
    \begin{itemize}
      \item \texttt{Lattice(std::map<std::string, std::pair<float, float>> lims, std::map<std::string, int> npts)}
        This is the constructor for the Lattice class. 
        The lims map it takes in is constructed from the lims input parameters.
        The string key of the lims map should be x, y, or z, and its value 
        should be a pair with the first element being the minimum spatial 
        position in the lattice, and the second element being the maximum.
        The npts input map has a string key of x, y, or z, and an integer 
        defining the number of points in that dimension.
        This setup is currently specific only to a simple-cubic geometry in
        three dimensions. 
        The return value is a Lattice object.
      \item \texttt{int getNumSites()}
        Returns the number of sites in the lattice. 
        Once higher dimensions are supported, this will be a vector npts.
      \item \texttt{float operator[](const int index) const}
        This overloaded operator returns the spatial position contained in the
        lattice at the given index. 
        Eventually this will support three dimensions:
        [x\_ind, y\_ind, z\_ind].
    \end{itemize}
  \item[\texttt{\textbf{Configuration}}]
    Stores information about the current configuration including a map from taus 
    to Bonds, and the tolerance used to determine uniqueness of a taus. \\
    \textbf{Data Members}
    \begin{itemize}
      \item \texttt{double tolerance}
        The tolerance/precision to use when accessing bonds in the 
        configuration by their tau key. 
        The default tolerance is 1e-5. 
      \item \texttt{std::map<double, Bond> bonds}
        A map from the "imaginary" time tau, a double, to a Bond type. 
    \end{itemize}
    \textbf{Methods}
    \begin{itemize}
      \item \texttt{Configuration(double tol)}
        This is the constructor for the Configuration class.
        It takes in a tolerance parameter and returns an empty Configuration 
        object.
      \item \texttt{void addBond(double tau, Bond\& newBond)}
        Adds a bond with a given tau and Bond to the configuration.
      \item \texttt{void addBonds(std::vector<double> taus, std::vector<Bond> newBonds)}
        Adds multiple bonds to a configuration at once using a vector of taus
        and a vector of Bonds.
        Calls the addBond method.
      \item \texttt{void delBond(double tau)}
        Deletes a Bond from the configuration by its tau. 
      \item \texttt{void delBonds()}
        Deletes all bonds from a configuration. 
      \item \texttt{const Bond\& getBond(double tau) const}
        Returns a Bond object from its tau value. 
      \item \texttt{const std::map<double, Bond>\& getBonds() const}
        Returns the bonds map, which is a map of all Bonds in the Configuration.
      \item \texttt{int getNumBonds() const}
        Returns the number of Bonds (number of unique taus) in the Configuration. 
      \item \texttt{bool operator==(const Configuration\& other) const}
        An overloaded equality operator which tests the equality of two 
        Configurations by testing the equality of the number of bonds in each
        Configuration and the equality of each Bond associated with each tau.
      \item \texttt{bool operator!=(const Configuration\& other) const}
        An overloaded inequality operator which calls the overloaded equality 
        operator to determine if two Configurations are not equivalent.
      \item \texttt{friend std::ostream\& operator<<(std::ostream\& os, const Configuration\& configuration)}
        An overloaded output operator which facilitates easy viewing of a 
        Configuration object.
      \item \texttt{private double truncateToTolerance(double key) const}
        A function which rounds a double to a given tolerance. 
        The tolerance should already be stored in the Configuration before this
        method is ever called.
    \end{itemize}
\end{description}


% Hamiltonian Representation in Code
\subsection{Hamiltonian Representation}


% Fermion Bag Algorithm Implementation (where it happens, algorithmic flow, etc.)
\subsection{Fermion-Bag Algorithm}


% Measurement of Observables
\subsection{Measurement of Observables}