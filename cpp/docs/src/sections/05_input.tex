% ============================================================
% Section 05 - Input Parameters
% ============================================================

\section{Input Parameters}
\label{sec:input}
\sloppy


% Format
\subsection{Input File Format}
The input files for this code are YAML files, so parameter specifications are
in the form of key-value pairs. 
Variables are often nested under a few keys for increased clarity.
There are four primary categories of variables that can be set in these input 
files: control, lattice, configuration, and constants.


% Input Parameter Descriptions
\subsection{Input Parameters}
\begin{itemize}
  \item \texttt{control}
    Control parameters are any variables determining the overall execution of 
    the code. 
    \begin{itemize}
      \item \texttt{hamil_model}
        The bond switching algorithm to use in the relaxation process.
        Currently, the only option is "random", but others will be introduced
        as development proceeds.
        \begin{itemize}
          \item \texttt{random}
            This hamiltonian model randomly adds or removes a bond at a given
            step according to some probability of addition or removel (name?). 
            It starts with an empty starting configuration and proceeds with
            additions and removals.
            The location of the bond is random, the tau value is random between
            0 and beta, and the number of sites in the bond is randomly chosen
            according to bond_type_props.
            This proceeds for nmoves iterations.
            For more information, see \ref{sec:physics_implementations}.
        \end{itemize}
      \item \texttt{nmoves}
        The number of times the configuration should be updated.
    \end{itemize}
  \item \texttt{lattice}
    Lattice parameters include all variables needed to fully determine the 
    lattice structure. 
    \begin{itemize}
      \item \texttt{type}
        The Bravais lattice type. 
        The only supported type is "simple-cubic" in one dimension at the 
        moment. 
        Support will be added for up to three dimensions and a variety of 
        Bravais lattice types.
      \item \texttt{a, b, c}
        The axial lengths of the Bravais lattice in Bohr.
        If only a is supplied, the lattice will be one dimensional.
        If a and b are supplied it will be two, and if all three are 
        supplied it will be three dimensional. 
      \item \texttt{alpha, beta, gamma}
        The angles completing the Bravais lattice definition in radians. 
        If only a is supplied, these angles will be ignored.
        If a and b are supplied, only alpha is required. 
        If a, b, and c are supplied, all angles are required.
      \item \texttt{lims}
        The lims section is for defining information on the overall spatial 
        extent of the lattice. 
        \begin{itemize}
          \item \texttt{x, y, z}
            Variables x, y, and x are subcategories of the lims section.
            If a is supplied, x must be set, etc. 
            \begin{itemize}
              \item \texttt{min}
                The minimum value in Bohr that the lattice should extend to in
                the given dimension.
              \item \texttt{max\_factor}
                Instead of setting a maximum limit for the given dimension, you
                choose the maximum number of multiples of the unit cell as a 
                way of specifying the upper limit.  
                This parameter, along with "base", determines the upper limit 
                of the lattice.
              \item \texttt{base}
                The multiplier which acts as the "unit" that max\_factor is 
                multiplied by to determine the upper boundary. 
            \end{itemize}
        \end{itemize}
    \end{itemize}
  \item \texttt{configuration}
    Configuration parameters fully determine the attributes of configurations
    throughout the progression of the code. 
    For definitions of configurations and related concepts, see Sections
    \ref{sec:theory} and \ref{sec:physics_implementations}.
    \begin{itemize}
      \item \texttt{float\_tol}
        The float tolerance determines how precise the tau variable needs to be
        when retrieving the bond at a given tau. 
        Bonds are stored in a hashmap, mapping from the tau float variable to a
        bond object. 
        If no floating point tolerance is specified, machine precision 
        discrepancies could contribute to the code not recognizing a given tau. 
        The default float tolerance is 1e-5. 
      \item \texttt{beta}
        Beta, or inverse temperature, is the maximum value that the tau 
        variable in the code can take.
        The tau variable is an "imaginary" time variable.
      \item \texttt{bond\_type\_props}
        Bonds in the code can be made up of one or more adjacent lattice sites.
        The purpose of this parameter is to specify which bond sizes should be
        included in the simulation, and the relative proportions of each size.
        Bond sizes are given as keys under the bond\_type\_props section.
        The relative proportions are given as the values corresponding to each 
        key.
        The proportions can be given as floats or integers. 
        Renormalization is done in the code.
    \end{itemize}
  \item \texttt{constants}
    This parameter section is the place to define useful constants that may be
    needed to fully specify the input. 
    For example, Bravais angles for a simple-cubic lattice need to be set to 
    pi/2.
    So, pi/2 would be a useful constant to store here.
    The code does not rely on these user constants to perform its calculations.
    Where those are needed, they are defined in the code. 
\end{itemize}


% Environment Variable Descriptions
\subsection{Environment Variables}